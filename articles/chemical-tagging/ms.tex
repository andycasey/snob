%% This document is part of the snob project
%% Copyright 2017 the authors. All rights reserved.

\documentclass{aastex61}

% For revision history
\IfFileExists{vc.tex}{\input{vc.tex}}{
    \newcommand{\githash}{UNKNOWN}
    \newcommand{\giturl}{UNKNOWN}}

% Define commands
\newcommand{\article}{\emph{Article}}
\newcommand{\acronym}[1]{{\small{#1}}}
\newcommand{\project}[1]{\textsl{#1}}

% Affiliation(s)
\newcommand{\moca}{
	\affil{School of Physics and Astronomy, Monash University, 
		   Melbourne, Clayton VIC 3800, Australia}}
\newcommand{\claytonfit}{
	\affil{Faculty of Information Technology, Monash University,
	       Melbourne, Clayton VIC 3800, Australia}}
\newcommand{\caulfieldfit}{
	\affil{Faculty of Information Technology, Monash University,
		   Melbourne, Caulfield East VIC 3145, Australia}}



\received{}
\revised{}
\accepted{}

\submitjournal{TBD} % ApJ, MNRAS, PASA?

\shorttitle{Chemical Tagging Using Minimum Message Length}
\shortauthors{Casey et al.}

\begin{document}

\title{Chemical Tagging Using Minimum Message Length}

\correspondingauthor{Andrew R. Casey}
\email{andrew.casey@monash.edu}

\author[0000-0003-0174-0564]{Andrew R. Casey}
\moca
\claytonfit

\author{(some order of:}

\author[0000-0003-2952-859X]{John Lattanzio}
\moca

\author[0000-0002-0583-5918]{David Dowe} 
\claytonfit

\author[0000-0002-1716-690X]{Aldeida Aleti}
\caulfieldfit

\author{)}

\author{others?}
 
\begin{abstract}
Chemical tagging seeks to identify co-natal stars by their present-day 
photospheric abundances, long after their phase space similarity is lost.
In principle the detailed chemical abundances of a transformative number of
stars could be used to reconstruct the evolution of the Milky Way, providing
insight on supernovae yields, galactic dynamics (e.g., constraints on radial 
mixing), the initial mass function, as well as the formation and destruction
of star clusters. Some progress has been made towards chemical tagging, but 
here we argue that even the \emph{extent} of the problem has not been realised.
We introduce the Real World problems associated with chemical tagging at scale,
and describe how currently considered methods will inevitably fail even under
the most unrealistically optimistic conditions. We introduce the minimum message
length to the astronomical community as a promising alternative to chemical
tagging, one which has the capability to address all ensivaged problems with
chemical tagging simultaneously. We perform chemical tagging experiments with
infinte Gaussian mixture models, and demonstrate using real and generated data 
that MML outperforms all other considered penalty techniques. 
\end{abstract}

\keywords{}

\section{Introduction} 
\label{sec:introduction}

\citet{Freeman_2002} introduced chemical tagging as an idea to identify groups of stars that formed together in the same gas cloud, which have now become physically separated in phase space.  This idea is attractive because the observable chemical abundances of stars remains largely unchanged throughout a star's lifetime, whereas phase space information is quickly damped through dynamical interactions. Some progress has been made towards chemical tagging in simplistic experiments, but the full extent of chemical tagging has yet to be realised. Indeed, in this \article\ we argue that the full extent \emph{of the problem} of chemical tagging has not even been realised yet.




\acknowledgments
Who?
Grants.

\software{astropy \citep{Robitaille:2013}, numpy, scipy, scikit-learn} 

\begin{thebibliography}{}

\bibitem[Astropy Collaboration et al.(2013)]{Robitaille:2013} Astropy Collaboration, Robitaille, T.~P., Tollerud, E.~J., et al.\ 2013, \aap, 558, A33 

\end{thebibliography}


\end{document}
