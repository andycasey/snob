\documentclass{article}

% if you need to pass options to natbib, use, e.g.:
% \PassOptionsToPackage{numbers, compress}{natbib}
% before loading nips_2017
%
% to avoid loading the natbib package, add option nonatbib:
% \usepackage[nonatbib]{nips_2017}

%\usepackage{nips_2017}

% to compile a camera-ready version, add the [final] option, e.g.:
\usepackage[final]{nips_2017}

\usepackage[utf8]{inputenc} % allow utf-8 input
\usepackage[T1]{fontenc}    % use 8-bit T1 fonts
\usepackage{hyperref}       % hyperlinks
\usepackage{url}            % simple URL typesetting
\usepackage{booktabs}       % professional-quality tables
\usepackage{amsfonts}       % blackboard math symbols
\usepackage{nicefrac}       % compact symbols for 1/2, etc.
\usepackage{microtype}      % microtypography

\title{
  Unsupervised learning of Gaussian mixture models in linear time
  using a variational Bayesian method
}

% The \author macro works with any number of authors. There are two
% commands used to separate the names and addresses of multiple
% authors: \And and \AND.
%
% Using \And between authors leaves it to LaTeX to determine where to
% break the lines. Using \AND forces a line break at that point. So,
% if LaTeX puts 3 of 4 authors names on the first line, and the last
% on the second line, try using \AND instead of \And before the third
% author name.

\author{
  Andrew R.~Casey\\
  Monash University\\
  \texttt{andrew.casey@monash.edu}\\
  \And 
  Aldeida Aleti\\
  Monash University\\
  \texttt{aldeida.aleti@monash.edu}\\
  \AND
  David Dowe\\
  Monash University\\
  \texttt{david.dowe@monash.edu}\\
  \And
  John C.~Lattanzio\\
  Monash University\\
  \texttt{john.lattanzio@monash.edu}\\
}

\begin{document}
% \nipsfinalcopy is no longer used

\maketitle

\begin{abstract}
  The appropriate number of components in a Gaussian mixture model is usually 
  determined by adopting a heuristic to penalize the number of free parameters.
  This approach unnecessarily separates model selection from parameter estimation,
  and requires a large set of mixtures with an increasing number of components
  to be optimized.
  We formulate an objective function for a finite Gaussian mixture model using
  Minimum Message Length (MML), and derive an expression to approximate that
  objective function for mixtures with any number of components. 
  These approximations provide a probability density for the evaluated objective
  function of future mixtures that have yet to be computed, allowing us to 
  \emph{jump} towards the optimal number of components.
  We show that our approach can simultaneously perform parameter estimation and 
  model selection for multivariate Gaussian mixtures in linear time.
\end{abstract}

\section{Introduction}
% Gaussian mixture modelling is a common problem.
% When the number of mixtures is unknown, the problem is normally 
% attacked by optimizing multiple different models, and performing
% model selection after-the-fact.
% Model selection is typically governed by a penalized heuristic,
% e.g., BIC, AIC, etc.

% Introducing MML
% Different approaches in terms of where to start from:
% Some things about how MML has been used to grow
% Some example where a mixture has been used where K=N,
% because merging/deleting components is computationally
% more efficient than splitting.

% In these approaches, the long-term memory of previously
% considered mixtures is usually discarded.
% Here we will outline an objective function using MML
% We will show that the difference in the message length
% between the current state and future unobserved states
% can be approximated.

\section{The objective function}


\section{Search strategy}

% Initialization

% Perturbation strategy with a twist


\section{Experimental results}

% 

% BAYES FTW

\section{Conclusion}


%\subsubsection*{Acknowledgments}

\section*{References}

References follow the acknowledgments. Use unnumbered first-level
heading for the references. Any choice of citation style is acceptable
as long as you are consistent. It is permissible to reduce the font
size to \verb+small+ (9 point) when listing the references. {\bf
  Remember that you can use a ninth page as long as it contains
  \emph{only} cited references.}
\medskip

\small

[1] Alexander, J.A.\ \& Mozer, M.C.\ (1995) Template-based algorithms
for connectionist rule extraction. In G.\ Tesauro, D.S.\ Touretzky and
T.K.\ Leen (eds.), {\it Advances in Neural Information Processing
  Systems 7}, pp.\ 609--616. Cambridge, MA: MIT Press.

[2] Bower, J.M.\ \& Beeman, D.\ (1995) {\it The Book of GENESIS:
  Exploring Realistic Neural Models with the GEneral NEural SImulation
  System.}  New York: TELOS/Springer--Verlag.

[3] Hasselmo, M.E., Schnell, E.\ \& Barkai, E.\ (1995) Dynamics of
learning and recall at excitatory recurrent synapses and cholinergic
modulation in rat hippocampal region CA3. {\it Journal of
  Neuroscience} {\bf 15}(7):5249-5262.

\end{document}
